% !TEX TS-program = pdflatex
% !TEX encoding = UTF-8 Unicode

% This is a simple template for a LaTeX document using the "article" class.
% See "book", "report", "letter" for other types of document.

\documentclass[11pt]{article} % use larger type; default would be 10pt

 \usepackage{cmap} % поис в PDF
 \usepackage[T2A]{fontenc} % кодировка
 \usepackage[utf8]{inputenc} % set input encoding (not needed with XeLaTeX) 
 \usepackage[english,russian]{babel} % локализация и переносы

%%% Examples of Article customizations
% These packages are optional, depending whether you want the features they provide.
% See the LaTeX Companion or other references for full information.

%%% PAGE DIMENSIONS
\usepackage{geometry} % to change the page dimensions
\geometry{a4paper} % or letterpaper (US) or a5paper or....
% \geometry{margin=2in} % for example, change the margins to 2 inches all round
% \geometry{landscape} % set up the page for landscape
%   read geometry.pdf for detailed page layout information

\usepackage{graphicx} % support the \includegraphics command and options
\graphicspath{{./images/}}

% \usepackage[parfill]{parskip} % Activate to begin paragraphs with an empty line rather than an indent

%%% PACKAGES
\usepackage{booktabs} % for much better looking tables
\usepackage{array} % for better arrays (eg matrices) in maths
\usepackage{paralist} % very flexible & customisable lists (eg. enumerate/itemize, etc.)
\usepackage{verbatim} % adds environment for commenting out blocks of text & for better verbatim
\usepackage{subfig} % make it possible to include more than one captioned figure/table in a single float
% These packages are all incorporated in the memoir class to one degree or another...

%%% HEADERS & FOOTERS
\usepackage{fancyhdr} % This should be set AFTER setting up the page geometry
\pagestyle{fancy} % options: empty , plain , fancy
\renewcommand{\headrulewidth}{0pt} % customise the layout...
\lhead{}\chead{}\rhead{}
\lfoot{}\cfoot{\thepage}\rfoot{}

%%% SECTION TITLE APPEARANCE
\usepackage{sectsty}
\allsectionsfont{\sffamily\mdseries\upshape} % (See the fntguide.pdf for font help)
% (This matches ConTeXt defaults)

%%% ToC (table of contents) APPEARANCE
\usepackage[nottoc,notlof,notlot]{tocbibind} % Put the bibliography in the ToC
\usepackage[titles,subfigure]{tocloft} % Alter the style of the Table of Contents
\renewcommand{\cftsecfont}{\rmfamily\mdseries\upshape}
\renewcommand{\cftsecpagefont}{\rmfamily\mdseries\upshape} % No bold!



%%% END Article customizations

\bibliographystyle{gost780u}
\bibliography{rpz}

%%% The "real" document content comes below...

\title{Разработка конструкции носимого тканевого оксиметра для контроля функционального состояния человека при физической нагрузке}
\author{Наумов Михаил Андреевич}
%\date{} % Activate to display a given date or no date (if empty),
         % otherwise the current date is printed 

\begin{document}
\maketitle
\newpage
\tableofcontents
\clearpage

\section{Введение}
Скелетная мускулатура образует самую крупную систему органов в человеческом организме [1]. Мышцы по большей части определяют силу, выносливость, скорость, ловкость, координацию и другие жизненно важные качества отдельно взятого человека, при этом данные показатели индивид может тренировать и улучшать с помощью методов физической подготовки. Для нормального функционирования этой системы необходимо постоянно снабжать её энергией и выводить продукты биохимических реакций. Выполнение этих задач обеспечивает мышечный кровоток. Во время физической нагрузки особенно важно отслеживать, как происходит перенос кислорода к различным тканям, поэтому измерение степени оксигенации гемоглобина в мышцах является информативным параметром.
В настоящее время отличительной чертой медицинской диагностики является продвижение в сторону неинвазивных, портативных, относительно недорогих методов и аппаратов [2]. Одним из таких методов является спектрофотометрия биологических тканей в красном и ближнем инфракрасном (К-БИК) диапазонах длин волн. Данный метод применяется для различных исследований, в том числе и в спортивной медицине.
В связи с вышеизложенной информацией целью данной работы является разработка конструкции носимого тканевого оксиметра для контроля функционального состояния человека при физической нагрузке.

\section{Основная часть}
\subsection{Проблема контроля функционального состояния человека при физической нагрузке}

%\qquad
Оценка функционального состояния организма человека является обязательной составляющей врачебного контроля за здоровьем лиц, занимающихся физической культурой и спортом. Функциональное тестирование в спортивной медицине основано на сопоставлении физиологических показателей организма в условиях мышечного покоя, дозированных и предельных физических нагрузок, а также восстановительного периода [3].

На основании результатов проведения функциональной пробы можно оценить большое количество параметров, в том числе функциональное состояние сердечно-сосудистой, дыхательной и других систем организма, физическую подготовленность к занятиям спортом и физической культурой, эффективность программ тренировок и реабилитации, физическую работоспособность и уровень подготовленности и адаптации к нагрузке. Также проба с физической нагрузкой позволяет проводить экспертизу профессиональной пригодности и выявлять различные предпаталогические состояния.

Проблема контроля функционального состояния при физической нагрузке имеет большое значение в спортивной медицине, рехабилитации и общей медицинской практике. В данном обзоре мы рассмотрим вопросы контроля кровотока и тканевой оксиметрии при физической нагрузке.

Контроль кровотока является одним из ключевых аспектов в оценке функционального состояния организма во время физической активности. Один из методов контроля кровотока - допплеровская ультразвуковая диагностика с применением допплеровской флоуметрии. Этот метод позволяет оценить локальные изменения кровотока в сосудах различных диаметров и определить его основные показатели, такие как скорость, объемный расход и сопротивление кровотока [4].


Тканевая оксиметрия, или измерение оксигенирования тканей, также является важным методом контроля функционального состояния организма. Она позволяет непосредственно измерять уровень оксигенирования тканей в реальном времени. Один из наиболее распространенных методов - мультиспектральная инфракрасная тканевая оксиметрия (MSOT), которая сочетает разные длины волн света и позволяет получить информацию о содержании оксигенированного и дезоксигенированного гемоглобина в тканях [5].



Таким образом, контроль кровотока и тканевой оксиметрии при физической нагрузке позволяет оценить функциональное состояние организма и тканей, а также выявить нарушения, связанные с оксигенированием и метаболической активностью.

\subsection{Список используемой литературы}

More text.

\end{document}
